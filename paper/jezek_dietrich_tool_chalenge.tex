
%% bare_conf.tex
%% V1.4b
%% 2015/08/26
%% by Michael Shell
%% See:
%% http://www.michaelshell.org/
%% for current contact information.
%%
%% This is a skeleton file demonstrating the use of IEEEtran.cls
%% (requires IEEEtran.cls version 1.8b or later) with an IEEE
%% conference paper.
%%
%% Support sites:
%% http://www.michaelshell.org/tex/ieeetran/
%% http://www.ctan.org/pkg/ieeetran
%% and
%% http://www.ieee.org/

%%*************************************************************************
%% Legal Notice:
%% This code is offered as-is without any warranty either expressed or
%% implied; without even the implied warranty of MERCHANTABILITY or
%% FITNESS FOR A PARTICULAR PURPOSE! 
%% User assumes all risk.
%% In no event shall the IEEE or any contributor to this code be liable for
%% any damages or losses, including, but not limited to, incidental,
%% consequential, or any other damages, resulting from the use or misuse
%% of any information contained here.
%%
%% All comments are the opinions of their respective authors and are not
%% necessarily endorsed by the IEEE.
%%
%% This work is distributed under the LaTeX Project Public License (LPPL)
%% ( http://www.latex-project.org/ ) version 1.3, and may be freely used,
%% distributed and modified. A copy of the LPPL, version 1.3, is included
%% in the base LaTeX documentation of all distributions of LaTeX released
%% 2003/12/01 or later.
%% Retain all contribution notices and credits.
%% ** Modified files should be clearly indicated as such, including  **
%% ** renaming them and changing author support contact information. **
%%*************************************************************************


% *** Authors should verify (and, if needed, correct) their LaTeX system  ***
% *** with the testflow diagnostic prior to trusting their LaTeX platform ***
% *** with production work. The IEEE's font choices and paper sizes can   ***
% *** trigger bugs that do not appear when using other class files.       ***                          ***
% The testflow support page is at:
% http://www.michaelshell.org/tex/testflow/



\documentclass[conference]{IEEEtran}
% Some Computer Society conferences also require the compsoc mode option,
% but others use the standard conference format.
%
% If IEEEtran.cls has not been installed into the LaTeX system files,
% manually specify the path to it like:
% \documentclass[conference]{../sty/IEEEtran}





% Some very useful LaTeX packages include:
% (uncomment the ones you want to load)


% *** MISC UTILITY PACKAGES ***
%
%\usepackage{ifpdf}
% Heiko Oberdiek's ifpdf.sty is very useful if you need conditional
% compilation based on whether the output is pdf or dvi.
% usage:
% \ifpdf
%   % pdf code
% \else
%   % dvi code
% \fi
% The latest version of ifpdf.sty can be obtained from:
% http://www.ctan.org/pkg/ifpdf
% Also, note that IEEEtran.cls V1.7 and later provides a builtin
% \ifCLASSINFOpdf conditional that works the same way.
% When switching from latex to pdflatex and vice-versa, the compiler may
% have to be run twice to clear warning/error messages.






% *** CITATION PACKAGES ***
%
%\usepackage{cite}
% cite.sty was written by Donald Arseneau
% V1.6 and later of IEEEtran pre-defines the format of the cite.sty package
% \cite{} output to follow that of the IEEE. Loading the cite package will
% result in citation numbers being automatically sorted and properly
% "compressed/ranged". e.g., [1], [9], [2], [7], [5], [6] without using
% cite.sty will become [1], [2], [5]--[7], [9] using cite.sty. cite.sty's
% \cite will automatically add leading space, if needed. Use cite.sty's
% noadjust option (cite.sty V3.8 and later) if you want to turn this off
% such as if a citation ever needs to be enclosed in parenthesis.
% cite.sty is already installed on most LaTeX systems. Be sure and use
% version 5.0 (2009-03-20) and later if using hyperref.sty.
% The latest version can be obtained at:
% http://www.ctan.org/pkg/cite
% The documentation is contained in the cite.sty file itself.






% *** GRAPHICS RELATED PACKAGES ***
%
\ifCLASSINFOpdf
  % \usepackage[pdftex]{graphicx}
  % declare the path(s) where your graphic files are
  % \graphicspath{{../pdf/}{../jpeg/}}
  % and their extensions so you won't have to specify these with
  % every instance of \includegraphics
  % \DeclareGraphicsExtensions{.pdf,.jpeg,.png}
\else
  % or other class option (dvipsone, dvipdf, if not using dvips). graphicx
  % will default to the driver specified in the system graphics.cfg if no
  % driver is specified.
  % \usepackage[dvips]{graphicx}
  % declare the path(s) where your graphic files are
  % \graphicspath{{../eps/}}
  % and their extensions so you won't have to specify these with
  % every instance of \includegraphics
  % \DeclareGraphicsExtensions{.eps}
\fi
% graphicx was written by David Carlisle and Sebastian Rahtz. It is
% required if you want graphics, photos, etc. graphicx.sty is already
% installed on most LaTeX systems. The latest version and documentation
% can be obtained at: 
% http://www.ctan.org/pkg/graphicx
% Another good source of documentation is "Using Imported Graphics in
% LaTeX2e" by Keith Reckdahl which can be found at:
% http://www.ctan.org/pkg/epslatex
%
% latex, and pdflatex in dvi mode, support graphics in encapsulated
% postscript (.eps) format. pdflatex in pdf mode supports graphics
% in .pdf, .jpeg, .png and .mps (metapost) formats. Users should ensure
% that all non-photo figures use a vector format (.eps, .pdf, .mps) and
% not a bitmapped formats (.jpeg, .png). The IEEE frowns on bitmapped formats
% which can result in "jaggedy"/blurry rendering of lines and letters as
% well as large increases in file sizes.
%
% You can find documentation about the pdfTeX application at:
% http://www.tug.org/applications/pdftex





% *** MATH PACKAGES ***
%
%\usepackage{amsmath}
% A popular package from the American Mathematical Society that provides
% many useful and powerful commands for dealing with mathematics.
%
% Note that the amsmath package sets \interdisplaylinepenalty to 10000
% thus preventing page breaks from occurring within multiline equations. Use:
%\interdisplaylinepenalty=2500
% after loading amsmath to restore such page breaks as IEEEtran.cls normally
% does. amsmath.sty is already installed on most LaTeX systems. The latest
% version and documentation can be obtained at:
% http://www.ctan.org/pkg/amsmath





% *** SPECIALIZED LIST PACKAGES ***
%
%\usepackage{algorithmic}
% algorithmic.sty was written by Peter Williams and Rogerio Brito.
% This package provides an algorithmic environment fo describing algorithms.
% You can use the algorithmic environment in-text or within a figure
% environment to provide for a floating algorithm. Do NOT use the algorithm
% floating environment provided by algorithm.sty (by the same authors) or
% algorithm2e.sty (by Christophe Fiorio) as the IEEE does not use dedicated
% algorithm float types and packages that provide these will not provide
% correct IEEE style captions. The latest version and documentation of
% algorithmic.sty can be obtained at:
% http://www.ctan.org/pkg/algorithms
% Also of interest may be the (relatively newer and more customizable)
% algorithmicx.sty package by Szasz Janos:
% http://www.ctan.org/pkg/algorithmicx




% *** ALIGNMENT PACKAGES ***
%
%\usepackage{array}
% Frank Mittelbach's and David Carlisle's array.sty patches and improves
% the standard LaTeX2e array and tabular environments to provide better
% appearance and additional user controls. As the default LaTeX2e table
% generation code is lacking to the point of almost being broken with
% respect to the quality of the end results, all users are strongly
% advised to use an enhanced (at the very least that provided by array.sty)
% set of table tools. array.sty is already installed on most systems. The
% latest version and documentation can be obtained at:
% http://www.ctan.org/pkg/array


% IEEEtran contains the IEEEeqnarray family of commands that can be used to
% generate multiline equations as well as matrices, tables, etc., of high
% quality.




% *** SUBFIGURE PACKAGES ***
%\ifCLASSOPTIONcompsoc
%  \usepackage[caption=false,font=normalsize,labelfont=sf,textfont=sf]{subfig}
%\else
%  \usepackage[caption=false,font=footnotesize]{subfig}
%\fi
% subfig.sty, written by Steven Douglas Cochran, is the modern replacement
% for subfigure.sty, the latter of which is no longer maintained and is
% incompatible with some LaTeX packages including fixltx2e. However,
% subfig.sty requires and automatically loads Axel Sommerfeldt's caption.sty
% which will override IEEEtran.cls' handling of captions and this will result
% in non-IEEE style figure/table captions. To prevent this problem, be sure
% and invoke subfig.sty's "caption=false" package option (available since
% subfig.sty version 1.3, 2005/06/28) as this is will preserve IEEEtran.cls
% handling of captions.
% Note that the Computer Society format requires a larger sans serif font
% than the serif footnote size font used in traditional IEEE formatting
% and thus the need to invoke different subfig.sty package options depending
% on whether compsoc mode has been enabled.
%
% The latest version and documentation of subfig.sty can be obtained at:
% http://www.ctan.org/pkg/subfig




% *** FLOAT PACKAGES ***
%
%\usepackage{fixltx2e}
% fixltx2e, the successor to the earlier fix2col.sty, was written by
% Frank Mittelbach and David Carlisle. This package corrects a few problems
% in the LaTeX2e kernel, the most notable of which is that in current
% LaTeX2e releases, the ordering of single and double column floats is not
% guaranteed to be preserved. Thus, an unpatched LaTeX2e can allow a
% single column figure to be placed prior to an earlier double column
% figure.
% Be aware that LaTeX2e kernels dated 2015 and later have fixltx2e.sty's
% corrections already built into the system in which case a warning will
% be issued if an attempt is made to load fixltx2e.sty as it is no longer
% needed.
% The latest version and documentation can be found at:
% http://www.ctan.org/pkg/fixltx2e


%\usepackage{stfloats}
% stfloats.sty was written by Sigitas Tolusis. This package gives LaTeX2e
% the ability to do double column floats at the bottom of the page as well
% as the top. (e.g., "\begin{figure*}[!b]" is not normally possible in
% LaTeX2e). It also provides a command:
%\fnbelowfloat
% to enable the placement of footnotes below bottom floats (the standard
% LaTeX2e kernel puts them above bottom floats). This is an invasive package
% which rewrites many portions of the LaTeX2e float routines. It may not work
% with other packages that modify the LaTeX2e float routines. The latest
% version and documentation can be obtained at:
% http://www.ctan.org/pkg/stfloats
% Do not use the stfloats baselinefloat ability as the IEEE does not allow
% \baselineskip to stretch. Authors submitting work to the IEEE should note
% that the IEEE rarely uses double column equations and that authors should try
% to avoid such use. Do not be tempted to use the cuted.sty or midfloat.sty
% packages (also by Sigitas Tolusis) as the IEEE does not format its papers in
% such ways.
% Do not attempt to use stfloats with fixltx2e as they are incompatible.
% Instead, use Morten Hogholm'a dblfloatfix which combines the features
% of both fixltx2e and stfloats:
%
% \usepackage{dblfloatfix}
% The latest version can be found at:
% http://www.ctan.org/pkg/dblfloatfix




% *** PDF, URL AND HYPERLINK PACKAGES ***
%
\usepackage{url}
% url.sty was written by Donald Arseneau. It provides better support for
% handling and breaking URLs. url.sty is already installed on most LaTeX
% systems. The latest version and documentation can be obtained at:
% http://www.ctan.org/pkg/url
% Basically, \url{my_url_here}.




% *** Do not adjust lengths that control margins, column widths, etc. ***
% *** Do not use packages that alter fonts (such as pslatex).         ***
% There should be no need to do such things with IEEEtran.cls V1.6 and later.
% (Unless specifically asked to do so by the journal or conference you plan
% to submit to, of course. )


% correct bad hyphenation here
\hyphenation{op-tical net-works semi-conduc-tor}


\begin{document}
%
% paper title
% Titles are generally capitalized except for words such as a, an, and, as,
% at, but, by, for, in, nor, of, on, or, the, to and up, which are usually
% not capitalized unless they are the first or last word of the title.
% Linebreaks \\ can be used within to get better formatting as desired.
% Do not put math or special symbols in the title.
\title{API Evolution Data Corpus and Tools Challenge}


% author names and affiliations
% use a multiple column layout for up to three different
% affiliations
\author{
	\IEEEauthorblockN{Kamil Jezek}
	\IEEEauthorblockA{
Department of Computer Science and Engineering \\
NTIS -- New Technologies for the Information Society \\
Faculty of Applied Sciences,
University of West Bohemia \\
Pilsen, Czech Republic \\
	kjezek@kiv.zcu.cz}

\and
	\IEEEauthorblockN{Jens Dietrich}
	\IEEEauthorblockA{School of Engineering and Advanced Technology\\
	Massey University\\
	Palmerston North, New Zealand\\
	J.B.Dietrich@massey.ac.nz} \\

}

% conference papers do not typically use \thanks and this command
% is locked out in conference mode. If really needed, such as for
% the acknowledgment of grants, issue a \IEEEoverridecommandlockouts
% after \documentclass

% for over three affiliations, or if they all won't fit within the width
% of the page, use this alternative format:
% 
%\author{\IEEEauthorblockN{Michael Shell\IEEEauthorrefmark{1},
%Homer Simpson\IEEEauthorrefmark{2},
%James Kirk\IEEEauthorrefmark{3}, 
%Montgomery Scott\IEEEauthorrefmark{3} and
%Eldon Tyrell\IEEEauthorrefmark{4}}
%\IEEEauthorblockA{\IEEEauthorrefmark{1}School of Electrical and Computer Engineering\\
%Georgia Institute of Technology,
%Atlanta, Georgia 30332--0250\\ Email: see http://www.michaelshell.org/contact.html}
%\IEEEauthorblockA{\IEEEauthorrefmark{2}Twentieth Century Fox, Springfield, USA\\
%Email: homer@thesimpsons.com}
%\IEEEauthorblockA{\IEEEauthorrefmark{3}Starfleet Academy, San Francisco, California 96678-2391\\
%Telephone: (800) 555--1212, Fax: (888) 555--1212}
%\IEEEauthorblockA{\IEEEauthorrefmark{4}Tyrell Inc., 123 Replicant Street, Los Angeles, California 90210--4321}}




% use for special paper notices
%\IEEEspecialpapernotice{(Invited Paper)}




% make the title area
\maketitle

% As a general rule, do not put math, special symbols or citations
% in the abstract
\begin{abstract}
Development of independently released software components is nowadays widely supported by tools.
Among other benefits, the tools help guarantee backward compatibility of new versions of components. 
In other words, non-breaking changes in components depend on ability of the tools to detect breaking changes.
But how the tools cope? In this work, we have selected tools that analyse syntactic compatibility of API changes
and tested how they perform. We provide results together with the test data used for the test. The data itself
are also valuable contribution of this work as they may be used for testing future tools.
\end{abstract}

% no keywords




% For peer review papers, you can put extra information on the cover
% page as needed:
% \ifCLASSOPTIONpeerreview
% \begin{center} \bfseries EDICS Category: 3-BBND \end{center}
% \fi
%
% For peerreview papers, this IEEEtran command inserts a page break and
% creates the second title. It will be ignored for other modes.
\IEEEpeerreviewmaketitle

\section{Introduction}

\section{Related Work}
In the technical domain, the term compatibility denotes\footnote{Source: the Merriam-Webster dictionary.} the ``ability to be used together'' and  ``designed to work with another device or system without modification''. Various definitions of compatibility related to software components exist, both in research \cite{canal2001compatibility,Belguidoum08formalization,taylor2009software,brada2011enhanced} % possibly stuckenholz06compatible, 
and technical \cite{forman1995release,osgi2010semvers,oracle2015compatibility} literature,
mostly dealing with the issue of either correct replacement or interoperability.

...

\section{Background: About Compatibility}

Consequence of a wrongly selected library is potential incompatibility with
its client. The notion of compatibility is complex
as every modification of a library may influence the way other libraries can use, interact, extend, observe or substitute it, in various ways.

The Java Language Specification formally defines a notion of binary compatibility
\cite[ch. 13]{Java7Spec} that is strictly defined with respect to the static analysis performed during linking. 
It significantly differs from the notion of source compatibility which 
is checked by the compiler as the consistency between a program and a library.
It also holds in Java that binary compatibility does not implies 
source compatibility and vice versa. For instance, specialising the return type 
of a method is source compatible for a client using this method, but not binary 
compatible. On the other hand, changing generic type parameters of a class is 
often binary compatible, but not source compatible. 

Reason for this discrepancy is in different rules enforced by the linker and the compiler. Whereas
the compiler performs certain type conversions, the linker requires exact type matching. For instance, the compiler performs boxing and unboxing to allow for seamless conversion between primitive and wrapper types, but wrapper and primitive types are not equivalent for the linker as the conversion is unknown for the linker.

When a program is built and deployed, a mixed notion of compatibility is used. As the program is compiled, the source compatibility with the libraries 
is checked by the compiler. In contrary, binary compatibility among the 
libraries is checked when the program is invoked.  It may result in situations where a system may be compiled but cannot run or vice-versa.

Compatibility also depends on how a library is used, 
i.e. if a library is used (invoked) only or also implemented (used for extension in sense of the object-oriented paradigm) by the client 
program. In other words, adding a method to a Java interface may be acceptable for the client invoking this interface but represent an 
incompatible change for someone implementing that interface as all methods must be implemented.
 
Compatibility also includes the semantics of programs. Simply said, a program is compatible with its libraries if they behave as expected, consequently causing the whole system behaving as expected. 

As handling the full compatibility from all directions is hard, simplified approaches are taken in practise. Most commonly compatibility is checked on libraries API where a library is understood as compatible when its API is compatible with clients. In other words, it is expected that a library does not change (worsen) its behaviour as long as API is not incompatibly changed. 

Although API may be enriched with sophisticated behaviour-related annotations such as pre/post conditions \cite{liskov87data}, practical applications still rely on syntactic API expressing type system, i.e. method signatures. Overview of possible API breaking changes is collected in  \cite{EvolvingJavaAPIs:2007}.
Benefit of checking only the syntax of API is that it may be easily automated and verified by tools. 


Belguidoum \cite{Belguidoum08formalization} distinguishes between \textit{vertical} and \textit{horizontal} compatibility which may be respectively paraphrased as backward and client-provider compatibility. Vertical compatibility is important for library vendors if they desire to produce new versions that are backward compatible with older ones and thus library update in a system will be smooth. On the other hand, horizontal compatibility is important for a system developer who checks if a system works with a selected set of libraries.

In this work, we propose a dataset simulating vertical compatibility (library updates) as well as horizontal compatibility (a client provider role) first.
Then we check how successfully is incompatibility detected by existing tools with concert to both source and binary compatibility.



\section{API Changes Data Corpus}

Test data proposed in this work composes a corpus of possible syntactic API changes, which model evolution and releases of consequent versions of a library. 
The data are separated into eight categories to handle complexity. The categories are: access modifiers, data types, exceptions, generics, inheritance, class members, other (non-access) modifiers and borderline (uncategorised) cases. Each category is filled with examples of API change with the desire to cover all possible changes. However, the data does not have to be complete as some cases did not have to be know in the time of building the corpus, or new cases may appear as the language evolve. To cope with this, the corpus is extendable and new examples may be added simply to respective directories.

The corpus is split into three directories: \texttt{lib-v1}, \texttt{lib-v2} and \texttt{client}. As the names suggest, the directories contain a first (original) version of an API, a second (evolved) version of an API and a client application, which invokes the API. The directories model real-life scenario where a client uses libraries, but the libraries in the corpus are simplistic and contain only API with dummy implementation. In real-word a more complex back-end would be invoked behind the API instead. The client application shows possible invocation of a library, thought it does not have to be exhaustive and more cases will actually exist.

Each library as well as the client hold a set of sub-directories, Java packages, representing concrete API evaluation example. The package names are constructed following way:
\begin{verbatim}
<category><element><change>
\end{verbatim} 

Where \texttt{category} is one of the eight categories, \texttt{element} is a representation of changed element and \texttt{change} describes content of the change. For instance, a case named \texttt{dataTypeClassFieldBoxing} means that a class field changed its data types and the concrete change was boxing.

The design based on the naming convention allows for extensibility of the corpus by simply adding new cases following the convention. In fact, the convention is not enforced, but recommended to keep order in the relatively big number of data.

The corpus contains an \texttt{ant} script in its root directory to build the source data into JAR files. It may be invoked simply by typing \texttt{ant} from the command-line.  The script output are three JAR files named the same way as the original source directories.

The whole structure of the corpus looks as follows (\texttt{<>} is shortcut for the \texttt{<category><element><change>} triplet described above):
\begin{verbatim}
<root>
  +- client/src/<>/Main.java
  +- lib-v1/src/lib/<>/<>.java
  +- lib-v2/src/lib/<>/<>.java
  build.xml
\end{verbatim}


Although the corpus described so far may be used as such, we provide additional meta-data with convenient information for third-party usage.
To stay up-to date when the corpus is extended, the meta-data must be generated before first usage.
A linux bash script named \texttt{compatibility.sh} and stored in the corpus root is provided to do this.
The script generates a simple CSV file with three columns: first one lists the name of a change as described above and second two columns inform respectively about source and binary compatibility. Value ``$1$'' is printed for a compatible change, ``$0$'' otherwise. 

The script works as follows: first, it automatically reads all changes stored in the \texttt{client,lib-v1,lib-v2} directories. Secondly, it tries to compile the client against the original library and then against the updated library. After that, it tries to invoke the client compiled against the old library with the updated library. The original library represents compatibility and thus the process should always succeed in the first step. On the other hand, it does not have to succeed in the second step as the updated library represents a potential incompatibility. Depending on the compilation and ivocation success, value ``1'' or ``0'' is printed in the CSV file to respective columns.


We make the corpus publicly available as a GitHub project hosted at: 
\begin{verbatim}
https://github.com/kjezek/
api-evolution-data-corpus/
\end{verbatim}

Following sub-sections detail API changes separated into categories. Short discussions is provided to overview why a category is signification.

\subsection{Access Modifiers}
- here and in following sections - nice examples of incompatibilities.
\subsection{Data Types}
\subsection{Exceptions}
\subsection{Generics}
\subsection{Inheritance}
\subsection{Members}
\subsection{Other Modifiers}
\subsection{Borderline Cases}


\section{Tools Challenge}

We searched for tools that are capable of discovering API syntactic backward compatibility and included them into this work. They are listed in Table \ref{tab:tools} together with basic information about their authors, current versions, licensing and basic usages. All these tools were benchmarked to find out how they cope with finding incompatibilities, results are provided below.

\begin{table*}[t]
  \centering
  \begin{tabular}{l | p{1cm} p{1cm} p{1cm} p{1cm} p{1cm} p{1cm} p{1cm} p{1cm} p{1cm} }
Tool  &  Clirr  &  Japicmp  &  japiChecker  &  JAPICC  &  Revapi  &  Sigtest  &  Japitools  &  Jour  &  JaCC	\\
\hline
\multicolumn{10}{|c|}{Basin info} \\
\hline
Author  &  Lars K�hne  &  Martin Mois  &  William Bernardet  &  Andrey Ponomarenko  &  Lukas Krejci  &  Oracle  &  Stuart Ballard  &  Vlad Skarzhevskyy  &  UWB	\\
License  &  LGPL  &  A2.0  &  A2.0  &  LGPL  &  A2.0  &  GPLv2�  &  GPL�  &  LGPL  &  ask	\\
Version  &  0.6.0  &  0.7.2  &  0.2.1  &  1.5  &  0.4.2  &  3.1  &  0.9.7  &  2.0.3  &  1.0.9	\\
Release  &  9/27/2005  &  3/20/2016  &  10/3/2015  &  4/8/2016  &  3/30/2016  &  4/8/2016  &  11/13/2007  &  12/12/2008  &  	\\
\hline
\multicolumn{10}{|c|}{Output} \\
\hline
TXT  &  yes  &  yes  &  yes  &    &  yes  &  yes  &  yes  &  yes  &  yes	\\
XML  &  yes  &  yes  &    &    &    &    &    &    &  	\\
HTML  &  yes  &  yes  &    &  yes  &    &    &    &    &  	\\
\hline
\multicolumn{10}{|c|}{Usage} \\
\hline
CLI  &  yes  &  yes  &  yes  &  yes  &  yes  &  yes  &  yes  &  yes  &  	\\
Maven  &  yes  &  yes  &  yes  &    &  yes  &  yes  &    &  yes  &  yes	\\
Ant  &  yes  &    &  yes  &    &  yes  &  yes  &    &    &  		\\
libray  &    &  yes  &    &    &    &    &    &    &  yes	\\
\end{tabular}

\caption{Tested Tools (GPL//LGPL = GNU GPL/LGPL, A2.0 = Apache 2.0)}
  \label{tab:tools}
\end{table*}

\subsection{Methodology}

The tools were tested using following approach: we generated the meta-data with incompatibility classification first, then we extracted only the lines that represent incompatibility. After that, we invoked all the tools and redirected their output to text files. In certain cases we removed lines from outputs representing a compatible change. Finally, we iterated the meta-data with incompatibilities and used string matching to find successfully detected incompatible changes in tools outputs. We collected the results in a CSV file with ``$1$'' meaning a correctly detected incompatibility and ``$0$'' meaning that a tool did not find the incompatibility.

The lines with compatible changes were removed from the meta-data because they represent only compatibility in the context of current client. In opposite, an incompatible usage not handled by the client may exist. If such a case were detected by a tool, it would not match with the expected compatibility and provide a wrong result. For this reason, we analysed only provably incompatible changes to prevent false negatives. 

The lines with compatible changes were removed from the tools outputs to prevent false positives. It could cause an incorrect string matching if a change were listed as compatible. This step differs for each tool. Some of the tools output only incompatibilities and do not have to be corrected (japichecker, japicc) while some have to. Usually such lines can be easily caught by a simple regular expression as they contain representative strings such as \texttt{UNCHANGED} (japicmp), \texttt{100\% Compatible} (japitool), \texttt{NON\_BREAKING} (revapi), \texttt{INFO} (clirr) etc.  

The whole process is automated and may be invoked by a bash script \texttt{./benchmark.sh}. The script prepares the meta-data, invokes the tools, corrects outputs and analyse results. It delegates invocation of the tools to the script \texttt{tools/run.sh}, which executes all tools one-by-one.

For instance the \texttt{run.sh} script contains following lines to invoke japicmp:
\begin{verbatim}
REPORTS=".reports"
java -jar japicmp/japicmp-0.7.2.jar \
  -o ../lib-v1.jar \
  -n ../lib-v2.jar \
  -a private > "$REPORTS"/japicmp.txt
  
grep -v '===  UNCHANGED' \
  "$REPORTS"/japicmp.txt > japicmp.txt.tmp 
mv japicmp.txt.tmp "$REPORTS"/japicmp.txt   
\end{verbatim}

The new tools may be added to the benchmark simply by adding invocable lines to the script. 
The script can contain any bash invocable lines that lead to execution of a tool and must be able to produce a textual output stored in the \texttt{tools/.reports}.

The structure of the corpus including the tools benchmark looks like:
\begin{verbatim}
<root>
  +- client/src/<>/Main.java
  +- lib-v1/src/lib/<>/<>.java
  +- lib-v2/src/lib/<>/<>.java
  +- tools/.reports
  +- tools/<tool>
  build.xml
  compatibility.sh
  benchmark.sh
\end{verbatim}

\subsection{Results}

The benchamrk has shown that the tools widely differ in their ability to find compatibility breaking changes. Results are provided in Table \ref{tab:result-categories}
as percentages of successfully detected compatibility breaking changes. They are separated to categories with a summary in the last row.

As it may be seen, the worst performing is \texttt{clirr}, which is however not actively developed since 2005. Evidently, \texttt{clirr} does not recognize generics and exceptions, but works well in other categories. It means that \texttt{clirr} may be still useful for detecting only binary incompatible changes.
Similar situation appears for \texttt{japicmp} showing a relatively poor result, which is however caused by unsporting generics and a few bugs in detecting modifiers. Otherwise, the tool is very reliable.

Interestingly, the second place is occupied by \texttt{japitool} which also seems to be no more developed since 2006, thought still available as part of Linux distributions (Debian/Linux in particular).

Tools  such as \texttt{japicc} or \texttt{revapi} showed overall better score, but they have several issues scattered among categories. For this reason, they may be less reliable in production than e.g. \texttt{clirr} as they can miss important issues. In contrast to \texttt{clirr}, both tools are still developed and may be thus improved in the future. 


\texttt{Sigtest} wins the benchmark as it is able to detect almost all problems. It fails only in two changes, detection of the removed \texttt{strictfp} modifier and addition of the \texttt{native} modifier, which are both binary incompatible. As they are very specific modifiers, we do not expect their frequent changes among library versions. For this reason, \texttt{sigtest} may serve as the most reliable tool from this benchmark.
  
Table \ref{tab:result-types} provides insight into results separated for source and binary incompatibilities. First line shows changes that are only source incompatible and binary compatible. The second line in contrast lists changes that are binary incompatible, but may be either source compatible or incompatible. 
We separated the data this way to test the tools specifically for source compatibility and for the rest. The table provides interesting results, the tools perform much better in detecting binary incompatibilities.  Worst in this category is \texttt{revapi} while most of the tools detected more than $90\%$ of issues. On the other hand, the tools lack ability to detect source incompatibility. The only actually reliable tools are \texttt{sigtest} and \texttt{japitool} that correctly recognised all source incompatibilities, partly useful is \texttt{revapi} with about 88\% of successful results.  Other tools detected only a small number of problems and cannot be recommended for industry level usage.

To summarize, the experiment shows that the most usable tool is \texttt{sigtest}, which is distributed as open-source and may be easily integrated into development process via CLI, Maven or Ant plugin. A small drawback may be that it produces only a TXT output, which makes it less convenient for third-party integrations. Furthermore, it was detected that other tools are in reality usable only for checking binary compatibility. Nonetheless, it may be sufficient in many scenarios as library updates are usually distributed in binary forms.  Hence, binary compatibility checking may help find the most unpredictable runtime failures caused by opaque third-party libraries. 
Although a source incompatible change may break a system as well, it is detectable by project build early in the development phase and thus less harmful.


\begin{table*}[t]
  \centering
  \begin{tabular}{l | r r r r r r r r r }
Category   &  clirr   &  jacc   &  japicc   &  japiChecker   &  japicmp   &  japitool   &  jour   &  revapi   &  sigtest  \\
\hline
Access Modifiers   &  100.00\%   &  100.00\%   &  83.33\%   &  100.00\%   &  100.00\%   &  100.00\%   &  83.33\%   &  83.33\%   &  100.00\% \\
Data Types   &  100.00\%   &  100.00\%   &  89.36\%   &  100.00\%   &  100.00\%   &  100.00\%   &  100.00\%   &  95.74\%   &  100.00\%  \\
Exceptions   &  0.00\%   &  0.00\%   &  100.00\%   &  100.00\%   &  100.00\%   &  100.00\%   &  100.00\%   &  71.43\%   &  100.00\%  \\
Generics   &  0.00\%   &  33.33\%   &  5.88\%   &  0.00\%   &  0.00\%   &  100.00\%   &  17.65\%   &  100.00\%   &  100.00\%  \\
Inheritance   &  71.43\%   &  100.00\%   &  71.43\%   &  85.71\%   &  100.00\%   &  100.00\%   &  100.00\%   &  42.86\%   &  100.00\%  \\
Members   &  100.00\%   &  100.00\%   &  84.21\%   &  89.47\%   &  100.00\%   &  100.00\%   &  84.21\%   &  42.11\%   &  100.00\%  \\
Other Modifiers   &  61.54\%   &  84.62\%   &  84.62\%   &  53.85\%   &  84.62\%   &  69.23\%   &  76.92\%   &  61.54\%   &  84.62\%  \\
Others   &  100.00\%   &  100.00\%   &  75.00\%   &  100.00\%   &  100.00\%   &  100.00\%   &  100.00\%   &  50.00\%   &  100.00\%  \\
\hline
Total  &  57.79\%   &  72.08\%   &  59.74\%   &  61.04\%   &  65.58\%   &  97.40\%   &  68.18\%   &  82.47\%   &  98.70\%  \\  
\end{tabular}
  \caption{Correctly Detected Incompatibilities}
  \label{tab:result-categories}
\end{table*}

\begin{table*}[t]
  \centering
  \begin{tabular}{l | r r r r r r r r r }
Type   &  clirr   &  jacc   &  japicc   &  japiChecker   &  japicmp   &  japitool   &  jour   &  revapi   &  sigtest  \\
\hline
Source 	& 13.24\%	&	41.18\%	&	25.00\%	&	20.59\%		&	25.00\%		&	100.00\%	&	38.24\%	&	88.24\%	&	100.00\%  \\
Binary	& 93.02\%	&	96.51\%	&	87.21\%	&	93.02\%		&	97.67\%		&	95.35\%		&	91.86\%	&	77.91\%	&	97.67\%		\\
%Both  	&  57.79\%   &  72.08\%   &  59.74\%   &  61.04\%   &  65.58\%   &  97.40\%   &  68.18\%   &  82.47\%   &  98.70\%  \\  
\end{tabular}
  \caption{Source vs Binary Incompatibilities}
  \label{tab:result-types}
\end{table*}


\subsection{Threats to Validity}

- data completeness
- bugs in data analysis

\section{Conclusion}
The conclusion goes here.




% conference papers do not normally have an appendix


% use section* for acknowledgment
\section*{Acknowledgment}


The authors would like to thank...


\bibliographystyle{plain}
\bibliography{references}

\end{document}


